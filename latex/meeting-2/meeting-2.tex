\documentclass[12pt]{article}

\usepackage{ulem}
\usepackage{float}
\usepackage{graphicx}
\usepackage{fancyhdr}
\usepackage{setspace}
\usepackage[hidelinks]{hyperref}
\usepackage[spanish]{babel}
\usepackage{mlmodern}
\usepackage{longtable}
\usepackage{tabularx}
\usepackage{multicol}
\usepackage[T1]{fontenc}
\usepackage[left=2cm, right=2cm, top=4cm, bottom=2.5cm, headheight=2cm, headsep=1cm]{geometry}
\renewcommand{\normalsize}{\fontsize{13}{18}\selectfont}

\pagestyle{fancy}
\fancyhead[C]
{
    \makebox[\textwidth]{
        \hspace{-0.2\textwidth}
        \begin{minipage}[c][2.5cm]{0.2\textwidth}
            \includegraphics[width=2.5cm]{logo}
        \end{minipage}
        \hspace{0.01\textwidth}
        \begin{minipage}[c][2.5cm]{0.6\textwidth}
            \centering
            \textbf{\LARGE Colegio Bautista Libertad}\\[0.2cm]
            \textbf{\Large Capacitación docente 2025}\\
            Contacto: soporte@cbl-edu.com
        \end{minipage}
    }
}
\fancyhead[L]{}
\fancyhead[R]{}
\fancyfoot[C]{Pág. \thepage}

\setlength{\parskip}{1.7mm}
\setlength{\parindent}{0pt}

\renewcommand{\headrulewidth}{0pt}

\usepackage{tocloft}
\hypersetup{colorlinks=true,linkcolor=black,urlcolor=blue,citecolor=blue}

\pagestyle{fancy}

\fancyhead[L]{Kenny J. Tinoco}
\fancyhead[R]{Ecuaciones Diofánticas}
%\fancyhead[C]{Nivel V}
\fancyfoot[C]{Pág. \thepage}

\fancypagestyle{first-page-style}
{
    \lhead{}
    \chead
    {
        \textbf
        {
            \LARGE Academia Sabatina de Jóvenes Talento \vspace{2mm}
        }
    }
    \rhead{}

    \addtolength{\topmargin}{0mm}
    \setlength{\headsep}{8mm}
}

\begin{document}
    \begin{center} \textbf
{
    \Large Ecuaciones Diofánticas \\ \vspace{2mm}Clase \#x
}
\end{center}

\begin{multicols}{2}
{
    \textbf{Encuentro:} x\\
    \textbf{Curso:} Ecuaciones Diofánticas\\
    \textbf{Fecha:} x de x de 2024\\
    \begin{flushright}
        \textbf{Nivel:} 5\\
        \textbf{Semestre:} II\\
        \textbf{Instructor:} Kenny Jordan Tinoco\\
        \textbf{Instructor Aux:} Gema Tapia
    \end{flushright}
}
\end{multicols}

\thispagestyle{first-page-style}

    \tableofcontents

    \section{Introducción}

    El presente documento expone los aspectos generales de uso del ``Sistema web para la gestión del Colegio Bautista Libertad'',
    este sistema tiene como objetivo digitalizar y optimizar los procesos internos de dicha institución.

    Este documento se centra en los conocimientos generales que el cuerpo docente debe conocer para el uso correcto de esta herramienta, 
    aborda puntos como: inicio de sesión en el sistema (correo y sistema de archivos), ingreso de calificaciones y navegación entre las distintas las opciones.

    En próximos encuentros se abordarán otros aspectos los cuales aún están en desarrollo y ajuste, previo a la utilización por parte de los usuarios.

    \section{Desarollo}

    En lo sucesivo nos referiremos a ``Sistema web para la gestión del Colegio Bautista Libertad''
    como ``WSMCBL'' por sus siglas en inglés\footnote{Web System for Managment of Colegio Bautista Libertad.} y
    a ``Colegio Bautista Libertad'' como ``CBL''.

    A continuación se detalla paso por paso cómo utilizar algunas de las opciones que ofrece esta herramienta.
    
    \subsection{Inicio de sesión}

    Como requisito para esta acción la administración de CBL deberá de suministrar un correo institucional y una contraseña para cada docente.

    \begin{enumerate}
        \item Ingresar a la página oficial de CBL por medio del enlace \href{www.cbl-edu.com}{cbl-edu.com}.
        \item Esta página muestra la información \textbf{pública} de WSMCBL, información relevante para el contacto y publicidad del colegio.
        \item Dirigirse a la parte superior derecha y seleccionar la opción ``WSM CBL'', este lo redirigirá hacía \href{wsm.cbl-edu.com}{wsm.cbl-edu.com}.
        \item Ingresar el correo institucional de CBL y su contraseña correcta.
        \item Se valida los datos y se ingresa a la parte \textbf{privada} de WSMCBL.
    \end{enumerate}

    Cabe resaltar que todo lo que el docente ingrese en WSMCBL será solo visible por la administración de CBL y otros docentes,
    es decir, no se reflejará en la parte pública del sistema.

    Por lo cual, es de suma importancia \textbf{no compartir sus credenciales} (correo o la contraseña) con ninguna persona para asegurar la integridad de los datos.

    En caso de perder de las credenciales puede contactarse al correo
    \href{mailto:soporte@cbl-edu.com}{soporte@cbl-edu.com}
    el cual le ayudará a recuperar el acceso de manera remota.
    En caso contrario puede recurrir a administración para que generen nuevas credenciales.


    \subsection{Registrar calificaciones}

    En WSMCBL los parciales del año lectivo puede estar activo o inactivo, correspondiendo a los períodos determinados por el MINED.
    Cada parcial tendrá un intervalo de tiempo conocido como \textbf{Registro de calificaciones} en el cual se podrá
    ingresar las calificaciones de los estudiantes, este intervalo será de dos semanas aproximadamente.

    Es decir, es requisito para esta acción que el registro de calificaciones esté activo para el parcial en curso,
    en caso contrario el docente \textbf{no podrá registrar ninguna calificación}, sin embargo, si podrá visualizar las calificaciones ya ingresadas.

    \begin{enumerate}
        \item Iniciar sesión en WSMCBL (seguir los pasos anteriores), debe estar en la página con enlace \href{wsm.cbl-edu.com}{wsm.cbl-edu.com}.
        \item Dirigirse a la parte izquierda de la pantalla y seleccionar la opción ``Académico'' y luego la opción ``Calificar''.
        \item Se muestra la lista de las secciones donde el docente imparte.
        \item Seleccionar la opción ``Calificar'' en una de las secciones.
        \item Se muestra la información de la sección, como la lista de estudiantes, la lista de asignaturas que el docente imparte y los campos para el ingreso de los datos.
        \item Ingresar la calificación estudiante por estudiante y al final de la tabla ingresar la calificación de conducta.
        \item Seleccionar guardar.
        \item Se registran las calificaciones de los estudiantes.
    \end{enumerate}

    Un punto importante a señalar es que la conducta ingresada será compartida.
    Si se da el caso de que el docente tiene más de una asignatura en la misma sección la conducta ingresada en el campo
    será la conducta de cada asignatura por separado.
    Es decir, si se tiene dos asignaturas y la calificación de conducta de un estudiante $A$ es 89, la conducta de las
    dos asignaturas también será de 89.
    Análogamente, sucederá con $n$ clases.

    Por otra parte, mientras el registro de calificaciones esté activo el docente podrá hacer correcciones a las calificaciones
    ya ingresadas, también, no es estrictamente necesario ingresar todas las calificaciones al inicio, si no se ingresan datos
    el sistema establece por defecto el valor de cero.

    \subsection{Parte docente}
    
    Se detalla el rol de los docentes en WSMCBL.
    Los docentes se encargarán de los siguientes puntos:
    \begin{itemize}
        \item Registrar de manera íntegra los datos de las calificaciones.
        \item Reportar cualquier error en información de los estudiantes: nombres, direcciones, números telefónicos, padecimientos, etc.
        \item Reportar faltas y retiros de los estudiantes.
        \item Reportar incumplimientos de docentes respecto a las calificaciones en las secciones guiadas.
        \item Utilizar los correos institucionales de CBL.
        \item Ocasionalmente corrección de los perfiles de los estudiantes.
    \end{itemize}

    Aspectos la administración se hará cargo y los docentes no harán:
    \begin{itemize}
        \item Asignar estudiantes a sus secciones guiadas.
        \item Corrección constante de los perfiles de estudiantes.
        \item Generación de listados oficiales por secciones.
        \item Generación de sabanas.
        \item Generación de boletas de calificaciones.
        \item Generación de boletínes.
        \item Generación de reportes de rendimiento.
        \item Generación de reportes de excelencia académica.
        \item Hojas de traslados.
        \item Algunos reportes del MINED.
    \end{itemize}

    Cabe mencionar que el éxito que WSMCBL pueda tener, recaerá en el uso correcto y constante de la información que se registre.

    \section{Conclusión}

    Debido a la forma de operar del Colegio Bautista Libertad este sistema trata de ser lo menos invasivo con respecto
    a los docentes, con lo cual esta herramienta solo necesita (a grandes rasgos) dos aspectos del cuerpo docente;
    primero, que provea de manera íntegra los datos de calificaciones para cada uno de sus estudiantes, y segundo, que
    vele por la correcta información de los estudiantes en el sistema.

    Con esto se espera digitalizar y optimizar los procesos internos del Colegio Bautista Libertad.

\end{document}