\documentclass[12pt]{article}

\usepackage{ulem}
\usepackage{float}
\usepackage{graphicx}
\usepackage{fancyhdr}
\usepackage{setspace}
\usepackage[hidelinks]{hyperref}
\usepackage[spanish]{babel}
\usepackage{mlmodern}
\usepackage{longtable}
\usepackage{tabularx}
\usepackage{multicol}
\usepackage[T1]{fontenc}
\usepackage[left=2cm, right=2cm, top=4cm, bottom=2.5cm, headheight=2cm, headsep=1cm]{geometry}
\renewcommand{\normalsize}{\fontsize{13}{18}\selectfont}

\pagestyle{fancy}
\fancyhead[C]
{
    \makebox[\textwidth]{
        \hspace{-0.2\textwidth}
        \begin{minipage}[c][2.5cm]{0.2\textwidth}
            \includegraphics[width=2.5cm]{logo}
        \end{minipage}
        \hspace{0.01\textwidth}
        \begin{minipage}[c][2.5cm]{0.6\textwidth}
            \centering
            \textbf{\LARGE Colegio Bautista Libertad}\\[0.2cm]
            \textbf{\Large Capacitación docente 2025}\\
            Contacto: soporte@cbl-edu.com
        \end{minipage}
    }
}
\fancyhead[L]{}
\fancyhead[R]{}
\fancyfoot[C]{Pág. \thepage}

\setlength{\parskip}{1.7mm}
\setlength{\parindent}{0pt}

\renewcommand{\headrulewidth}{0pt}

\usepackage{tocloft}
\hypersetup{colorlinks=true,linkcolor=black,urlcolor=blue,citecolor=blue}

\pagestyle{fancy}

\fancyhead[L]{Kenny J. Tinoco}
\fancyhead[R]{Ecuaciones Diofánticas}
%\fancyhead[C]{Nivel V}
\fancyfoot[C]{Pág. \thepage}

\fancypagestyle{first-page-style}
{
    \lhead{}
    \chead
    {
        \textbf
        {
            \LARGE Academia Sabatina de Jóvenes Talento \vspace{2mm}
        }
    }
    \rhead{}

    \addtolength{\topmargin}{0mm}
    \setlength{\headsep}{8mm}
}

\begin{document}
    \begin{center} \textbf
{
    \Large Ecuaciones Diofánticas \\ \vspace{2mm}Clase \#x
}
\end{center}

\begin{multicols}{2}
{
    \textbf{Encuentro:} x\\
    \textbf{Curso:} Ecuaciones Diofánticas\\
    \textbf{Fecha:} x de x de 2024\\
    \begin{flushright}
        \textbf{Nivel:} 5\\
        \textbf{Semestre:} II\\
        \textbf{Instructor:} Kenny Jordan Tinoco\\
        \textbf{Instructor Aux:} Gema Tapia
    \end{flushright}
}
\end{multicols}

\thispagestyle{first-page-style}

    \section{Introducción}

    El presente documento expone los aspectos generales de uso del ``Sistema web para la gestión del Colegio Bautista Libertad'',
    este sistema tiene como objetivo digitalizar y optimizar los procesos internos de dicha institución.

    Este documento se centra en los conocimientos generales que el cuerpo docente debe conocer para el uso correcto de esta herramienta, 
    aborda puntos como: inicio de sesión en el sistema (correo y sistema de archivos), visualización de la información del docente e
    ingreso de calificaciones.

    En próximos encuentros se abordarán otros aspectos los cuales aún están en desarrollo y ajuste, previo a la utilización por parte de los usuarios.

    \section{Desarollo}
    
    A continuación se detalla paso por paso como utilizar esta herramienta.
    Para abreviar, en lo siguiente nos referiremos a ``Sistema web para la gestión del Colegio Bautista Libertad''
    como ``WSMCBL'' por sus siglas en inglés\footnote{Web System for Managment of Colegio Bautista Libertad.} y
    a ``Colegio Bautista Libertad'' como ``CBL''.
    
    \subsection{Inicio de sesión}

    Como requisito para esta acción la administración de CBL deberá de suministrar un correo institucional y una contraseña para cada docente.

    \begin{enumerate}
        \item Ingresar a la página oficial de CBL por medio del enlace \href{www.cbl-edu.com}{cbl-edu.com}.
        \item Dirigirse a la parte superior derecha y seleccionar la opción ``WSM CBL'', este lo redirigirá hacía \href{wsm.cbl-edu.com}{wsm.cbl-edu.com}.
        \item Ingresar el correo institucional de CBL y su contraseña.
    \end{enumerate}


    \subsection{Registro de calificaciones}
    Como requisito para esta acción la administración de CBL deberá habilitar el registro de calificaciones para
    el parcial en curso, en caso contrario el docente no podrá registrar ninguna calificación, sin embargo, si podrá visualizar las calificaciones ya ingresadas.

    \begin{enumerate}
        \item Iniciar sesión en WSMCBL (seguir los pasos anteriores).
        \item Dirigirse a la parte izquierda de la pantalla y seleccionar la opción ``Académico'' y luego la opción ``Calificar''.
        \item Se muestra la lista de las secciones donde el docente imparte.
        \item Seleccionar la opción ``Calificar'' en una de las secciones.
        \item Se muestra la información de la sección, como la lista de estudiantes, la lista de asignaturas que el docente imparte en dicha sección y los campos para el ingreso de los datos.
        \item Ingresar la calificación estudiante por estudiante y al final de la tabla ingresar la calificación de conducta.
        \item Seleccionar guardar.
    \end{enumerate}

    Un punto importante a señalar es que la conducta ingresada en el caso de que un docente tenga más de una asignatura en la sección, será compartida por todas las asignaturas.

    \section{Conclusión}

    Debido a la forma de operar del Colegio Bautista Libertad este sistema trata de ser lo menos invasivo con respecto
    a los docentes, con lo cual esta herramienta solo necesita que el cuerpo docente provea de manera íntegra los datos de calificaciones para cada uno de sus estudiantes en tiempo y forma.
\end{document}